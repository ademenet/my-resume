\documentclass[10pt,a4paper,sans]{moderncv}
\usepackage{multicol}

\moderncvstyle{classic}
\moderncvcolor{black}

% Réglage de la largeur de la colonne
\setlength{\hintscolumnwidth}{4cm}

\usepackage[utf8]{inputenc} % encodage, à modifier selon vos habitudes
\usepackage[scale=0.8]{geometry} % pour régler les marges du CV les options habituelles de l'extension geometry peuvent s'appliquer ici
\usepackage{helvet} % pour utiliser la police helvetica par exemple.
\usepackage[french]{babel} % pour un document en français.

% Insertion des données personnelles
\name{Alain}{Demenet}
\title{Développeur – Formateur}
\address{3 rue du Landy}{92110 Clichy}{France}
\phone[mobile]{06~00~00~00~00}
% \phone[fixed]{01~01~88~33~55}
% \phone[fax]{02~11~22~33~44}
\email{alain.demenet@gmail.com}
\homepage{https://ademenet.github.io}
\social[linkedin]{alain-demenet}
% \social[twitter]{pierre.durand}
\social[github]{ademenet}
% \extrainfo{informations complémentaires.}
% \quote{Encore un titre}
\nopagenumbers % Évite d'afficher le compteur de page

% Insérer la photo
% \photo[64pt][0.4pt]{maphoto}

% Début du contenu
\begin{document}

\makecvtitle

\section{Expériences professionnelles}
    \cventry{juin 2020 • aujourd'hui}{Développeur}{Jedha Bootcamp}{Paris}{}
    {
        Développement de la plateforme d'apprentissage JULIE et d'outils de gestion internes.\\
        Technologies utilisées :
        {\setlength\multicolsep{0pt}
        \begin{multicols}{2}
            \begin{itemize}
                \item React, Next.js
                \item Python (multiples backends avec Flask et FastAPI)
                \item MongoDB
                \item AWS (Lambda, RDS, DynamoDB)
                \item Github Actions
            \end{itemize}
        \end{multicols}}
    }

    \cventry{novembre 2018 • juin 2020}{Data Scientist}{Freelance}{Paris}{}{}

    \cventry{juillet 2018 • janvier 2020}{Data Scientist}{QuantCube Technology}{Paris}{}
        {Développement d'algorithmes de segmentation d'images utilisant des solutions de deep learning.}

    \cventry{février 2016 • avril 2016}{Stage en deep learning}{Scortex}{Paris}{}
        {Développement d'algorithmes de détections d'objets et de vision par ordinateur.}

    \cventry{2013 • 2015}{Animateur}{écoles primaires}{Versailles}{}
        {Organisation et animation d'atelier scientifiques pour les enfants (robotique, informatique).}

\section{Formations}
    \cventry{2005 • 2009}{Licence Sciences et vie de la Terre}{Université Paris Diderot}{Paris}{}
        {}
    \cventry{2009 • 2011}{Master Médiations des connaissances environnementales}{Université de Versailles Saint-Quentin-en-Yvelines}{Versailles}{\textit{}}
        {}
    \cventry{2015 • 2018}{Formation architecte en technologie du Numérique}{École 42}{Paris}{}
        {
            L'école 42 est une école singulière sans professeurs ni cours. Nous y
            apprenons le développement informatique par l'intermédiaire de projets et du
            partage avec les pairs.
        }

\section{Compétences informatiques}
    % Possibilité de mettre les entrées en deux colonnes 
    \cvitem{Langages}{Python, Javascript, HTML, CSS, SQL, Bash, C/C++}
    \cvitem{Webdev}{React, Next.js}
    \cvitem{Machine Learning}{Pandas, Numpy, Scikit, Tensorflow}
    \cvitem{Other}{Git, Linux, Docker, AWS}

\section{Langues}
    % Possibilité d'insérer des commentaires dans les entrées
    \cvdoubleitem{Anglais}{Lu, parlé, écrit}{Espagnol}{Notions}

\section{Centres d'intérêts}
    \cvdoubleitem{Photo argentique}{Dévelopemment et tirage photos}{Wushu}{Arts martiaux chinois}
    \cvdoubleitem{Trompette}{Fanfares}{}{}

\end{document}